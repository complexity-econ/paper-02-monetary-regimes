\documentclass[11pt,a4paper]{article}

% Fonts and encoding
\usepackage{fontspec}
\usepackage{polyglossia}
\setmainlanguage{english}

% Page geometry
\usepackage[margin=2.5cm]{geometry}

% Math
\usepackage{amsmath}
\usepackage{amssymb}

% Graphics
\usepackage{graphicx}
\usepackage{float}
\graphicspath{{../figures/}}

% Tables
\usepackage{booktabs}
\usepackage{tabularx}

% Hyperlinks
\usepackage{hyperref}
\hypersetup{
    colorlinks=true,
    linkcolor=blue,
    citecolor=blue,
    urlcolor=blue
}

% Bibliography
\usepackage[backend=bibtex,style=authoryear,sorting=nyt,maxnames=3]{biblatex}
\addbibresource{references.bib}

% Line spacing
\usepackage{setspace}
\onehalfspacing

% Section formatting
\usepackage{titlesec}
\titleformat{\section}{\large\bfseries}{\thesection.}{0.5em}{}
\titleformat{\subsection}{\normalsize\bfseries}{\thesubsection.}{0.5em}{}

% Headers
\usepackage{fancyhdr}
\pagestyle{fancy}
\fancyhf{}
\fancyhead[L]{\small Monetary Regime and the Acceleration Paradox}
\fancyhead[R]{\small\thepage}

\begin{document}

\begin{center}
{\LARGE\bfseries Monetary Regime and the Acceleration Paradox:\\
PLN vs EUR Counterfactual in a Small Open Economy}

\vspace{0.8cm}

{\large Mateusz Maciaszek}

\vspace{0.3cm}

{\small February 2026}

\vspace{0.5cm}

\begin{minipage}{0.85\textwidth}
\small\textbf{Abstract.}
\autocite{maciaszek2026acceleration} demonstrated a phase transition in technology adoption at UBI~=~2,000~PLN, driven by the interaction of fiscal stimulus with endogenous monetary policy.
This paper tests whether the acceleration paradox requires monetary sovereignty by comparing two regimes: (1)~the status quo with NBP's Taylor rule and floating PLN/EUR exchange rate, and (2)~a counterfactual where Poland is a Eurozone member with exogenous ECB policy and fixed exchange rate.
Using the same SFC-ABM framework (10,000 firms, 6 sectors, Watts-Strogatz network, 120 months) across 1,260 Monte Carlo simulations (21 UBI levels $\times$ 2 regimes $\times$ 30 seeds), we find that [RESULTS TO BE FILLED AFTER SIMULATION].
The results formalize the role of Mundell's impossible trinity in the acceleration paradox mechanism.

\vspace{0.3cm}
\textbf{Keywords:} agent-based model, stock-flow consistent, monetary regime, Eurozone, UBI, automation, complexity economics, Mundell trilemma

\vspace{0.3cm}
\textbf{JEL:} E17, E42, E52, E58, F33, O33
\end{minipage}
\end{center}

\vspace{1cm}

% ═══════════════════════════════════════════════════════════════
\section{Introduction}
\label{sec:intro}

The acceleration paradox --- the finding that universal basic income (UBI) catalyzes rather than responds to automation --- was established in \autocite{maciaszek2026acceleration} using a stock-flow consistent agent-based model calibrated to the Polish economy.
The mechanism depends critically on two channels:
\begin{enumerate}
    \item \textbf{Interest rate channel}: The central bank (NBP) raises rates in response to UBI-driven inflation, tightening credit conditions and forcing firms toward capital-light automation.
    \item \textbf{Exchange rate channel}: Currency depreciation from the trade deficit raises import costs, further squeezing firm margins and accelerating the technology adoption decision.
\end{enumerate}

Both channels presuppose monetary sovereignty --- an independent central bank with a floating exchange rate.
If Poland were a Eurozone member, the ECB would set interest rates based on Eurozone-wide inflation (largely unaffected by Polish domestic policy), and there would be no exchange rate adjustment mechanism.

This raises a fundamental question: \textbf{Is monetary sovereignty a prerequisite for the acceleration paradox?}

We test this by running the identical SFC-ABM under two monetary regimes and comparing the resulting bifurcation diagrams, adoption dynamics, and welfare outcomes.

% ═══════════════════════════════════════════════════════════════
\section{Model Extension}
\label{sec:model}

We extend the model from \autocite{maciaszek2026acceleration} with a single structural change: the monetary regime.
All other parameters (firm heterogeneity, sector calibration, network topology, fiscal policy) remain identical.

\subsection{PLN Regime (Baseline)}

The NBP sets the reference rate using a Taylor rule:
\begin{equation}
    r_t^{\text{NBP}} = \rho \cdot r_{t-1} + (1-\rho)\left[ r^* + \alpha \max(0, \pi_t - \bar\pi) + \beta \max(0, \Delta e_t) \right]
\end{equation}
where $\rho = 0.70$ (inertia), $r^* = 4\%$ (neutral), $\alpha = 1.5$, $\beta = 0.8$, $\bar\pi = 2.5\%$.
The exchange rate adjusts endogenously via balance-of-payments:
\begin{equation}
    \Delta e_t = -\gamma \cdot \frac{\text{BoP}_t}{\text{GDP}_t}, \quad \gamma = 0.02
\end{equation}

\subsection{EUR Regime (Counterfactual)}

The ECB sets rates based on Eurozone-wide inflation $\pi^{\text{EZ}}$, which is exogenous to Poland:
\begin{equation}
    r_t^{\text{ECB}} = \rho^{\text{ECB}} \cdot r_{t-1} + (1-\rho^{\text{ECB}})\left[ r^{*,\text{ECB}} + \alpha^{\text{ECB}} \max(0, \pi^{\text{EZ}} - \bar\pi^{\text{ECB}}) \right]
\end{equation}
where $\rho^{\text{ECB}} = 0.85$, $r^{*,\text{ECB}} = 2.5\%$, $\alpha^{\text{ECB}} = 1.5$, $\bar\pi^{\text{ECB}} = 2.0\%$, $\pi^{\text{EZ}} = 2.0\%$ (constant).

The exchange rate is fixed at the conversion rate (4.33 PLN/EUR), eliminating:
\begin{itemize}
    \item Exchange rate pass-through to inflation
    \item Interest rate parity capital flows
    \item Export competitiveness via depreciation
\end{itemize}

\subsection{Mechanism Hypothesis}

In the PLN regime, UBI triggers a feedback loop:
$\text{UBI} \to \pi\!\uparrow \to r^{\text{NBP}}\!\uparrow \to \text{credit}\!\downarrow \to \text{automation}\!\uparrow \to \text{tech deflation}$.

In the EUR regime, this loop is broken: Polish inflation does not affect the ECB rate, so the credit channel remains loose regardless of domestic conditions.
We hypothesize that this fundamentally alters the system's phase portrait.

% ═══════════════════════════════════════════════════════════════
\section{Results}
\label{sec:results}

[TO BE COMPLETED AFTER RUNNING 1,260 SIMULATIONS]

\subsection{Regime Comparison at Key UBI Levels}

% Table placeholder
\begin{table}[H]
\centering
\caption{Terminal values at M120: PLN vs EUR (mean $\pm$ std, $N=30$ seeds)}
\label{tab:regime_comparison}
\begin{tabular}{llrrr}
\toprule
\textbf{Metric} & \textbf{Regime} & \textbf{UBI = 0} & \textbf{UBI = 2,000} & \textbf{UBI = 3,000} \\
\midrule
Adoption (\%)    & PLN & -- & -- & -- \\
                 & EUR & -- & -- & -- \\
Inflation (\%)   & PLN & -- & -- & -- \\
                 & EUR & -- & -- & -- \\
Unemployment (\%) & PLN & -- & -- & -- \\
                 & EUR & -- & -- & -- \\
\bottomrule
\end{tabular}
\end{table}

\subsection{Bifurcation Diagram Comparison}

% Figure placeholder
% \begin{figure}[H]
%     \centering
%     \includegraphics[width=\textwidth]{p02_bifurcation_comparison.png}
%     \caption{Bifurcation diagram: PLN vs EUR across 21 UBI levels.}
%     \label{fig:bifurcation}
% \end{figure}

\subsection{Critical Point Shift}

[Analysis of whether the critical point (max variance of adoption) shifts between regimes]

\subsection{Welfare Analysis}

% \begin{figure}[H]
%     \centering
%     \includegraphics[width=\textwidth]{p02_welfare_comparison.png}
%     \caption{Welfare comparison: real consumption and Gini coefficient.}
%     \label{fig:welfare}
% \end{figure}

% ═══════════════════════════════════════════════════════════════
\section{Discussion}
\label{sec:discussion}

\subsection{Mundell's Impossible Trinity}

[Discussion of how results relate to the impossible trinity: fixed exchange rate, free capital flows, independent monetary policy --- choose two]

\subsection{Internal Devaluation Trap}

[Discussion of how EUR regime forces adjustment through wages/prices rather than exchange rate, and whether this creates a deflation trap]

\subsection{Policy Implications}

[Implications for the Polish Euro adoption debate]

% ═══════════════════════════════════════════════════════════════
\section{Conclusion}
\label{sec:conclusion}

[TO BE COMPLETED]

% ═══════════════════════════════════════════════════════════════
\printbibliography

\end{document}
