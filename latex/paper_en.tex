\documentclass[11pt,a4paper]{article}

% Fonts and encoding
\usepackage{fontspec}
\usepackage{polyglossia}
\setmainlanguage{english}

% Page geometry
\usepackage[margin=2.5cm]{geometry}

% Math
\usepackage{amsmath}
\usepackage{amssymb}

% Graphics
\usepackage{graphicx}
\usepackage{float}
\graphicspath{{../figures/}}

% Tables
\usepackage{booktabs}
\usepackage{tabularx}

% Hyperlinks
\usepackage{hyperref}
\hypersetup{
    colorlinks=true,
    linkcolor=blue,
    citecolor=blue,
    urlcolor=blue
}

% Bibliography
\usepackage[backend=bibtex,style=authoryear,sorting=nyt,maxnames=3]{biblatex}
\addbibresource{references.bib}

% Line spacing
\usepackage{setspace}
\onehalfspacing

% Section formatting
\usepackage{titlesec}
\titleformat{\section}{\large\bfseries}{\thesection.}{0.5em}{}
\titleformat{\subsection}{\normalsize\bfseries}{\thesubsection.}{0.5em}{}

% Headers
\usepackage{fancyhdr}
\pagestyle{fancy}
\fancyhf{}
\fancyhead[L]{\small Monetary Regime and the Acceleration Paradox}
\fancyhead[R]{\small\thepage}

\begin{document}

\begin{center}
{\LARGE\bfseries Monetary Regime and the Acceleration Paradox:\\
PLN vs EUR Counterfactual in a Small Open Economy}

\vspace{0.8cm}

{\large Mateusz Maciaszek}

\vspace{0.3cm}

{\small February 2026}

\vspace{0.5cm}

\begin{minipage}{0.85\textwidth}
\small\textbf{Abstract.}
\autocite{maciaszek2026acceleration} demonstrated that universal basic income (UBI) can catalyze automation in a small open economy through monetary feedback channels.
This paper tests whether the acceleration paradox requires monetary sovereignty by comparing two regimes: (1)~the status quo with NBP's Taylor rule and floating PLN/EUR exchange rate, and (2)~a counterfactual where Poland is a Eurozone member with exogenous ECB policy and fixed exchange rate.
Using the same SFC-ABM framework (10,000 firms, 6 sectors, Watts-Strogatz network, 120 months) across 1,260 Monte Carlo simulations (21 UBI levels $\times$ 2 regimes $\times$ 30 seeds), we find that the two regimes produce fundamentally different dynamics.
Under PLN, the NBP raises rates in response to UBI-driven inflation, stabilizing employment (${\sim}8\%$ unemployment) but eroding real consumption as UBI rises.
Under EUR with Stability and Growth Pact (SGP) fiscal constraints, the ECB does not respond to Polish inflation and the SGP progressively eliminates UBI spending as the fiscal position deteriorates.
The economy enters a permanent deflationary trap (${\sim}{-}25\%$) with mass unemployment (${\sim}78\%$); a transient UBI burst in the first months raises adoption to ${\sim}19\%$ (vs ${\sim}13\%$ under PLN), but the SGP caps effective UBI at ${\sim}1{,}200$~PLN regardless of the legislated amount, producing identical outcomes for all legislated UBI $\geq 1{,}250$~PLN.
The critical point (maximum adoption variance) shifts from UBI$=500$ (PLN) to UBI$=1{,}000$ (EUR), confirming that monetary sovereignty reshapes the system's phase portrait.
These results formalize the role of Mundell's impossible trinity in the acceleration paradox mechanism and reveal that Eurozone fiscal constraints eliminate the policy space needed to exploit the paradox.

\vspace{0.3cm}
\textbf{Keywords:} agent-based model, stock-flow consistent, monetary regime, Eurozone, UBI, automation, complexity economics, Mundell trilemma

\vspace{0.3cm}
\textbf{JEL:} E17, E42, E52, E58, F33, O33
\end{minipage}
\end{center}

\vspace{1cm}

% ═══════════════════════════════════════════════════════════════
\section{Introduction}
\label{sec:intro}

The acceleration paradox --- the finding that universal basic income (UBI) catalyzes rather than responds to automation --- was established in \autocite{maciaszek2026acceleration} using a stock-flow consistent agent-based model calibrated to the Polish economy.
The mechanism depends critically on two channels:
\begin{enumerate}
    \item \textbf{Interest rate channel}: The central bank (NBP) raises rates in response to UBI-driven inflation, tightening credit conditions and forcing firms toward capital-light automation.
    \item \textbf{Exchange rate channel}: Currency depreciation from the trade deficit raises import costs, further squeezing firm margins and accelerating the technology adoption decision.
\end{enumerate}

Both channels presuppose monetary sovereignty --- an independent central bank with a floating exchange rate.
If Poland were a Eurozone member, the ECB would set interest rates based on Eurozone-wide inflation (largely unaffected by Polish domestic policy), and there would be no exchange rate adjustment mechanism.

This raises a fundamental question: \textbf{Is monetary sovereignty a prerequisite for the acceleration paradox?}

We test this by running the identical SFC-ABM under two monetary regimes and comparing the resulting bifurcation diagrams, adoption dynamics, and welfare outcomes.

% ═══════════════════════════════════════════════════════════════
\section{Model Extension}
\label{sec:model}

We extend the model from \autocite{maciaszek2026acceleration} with a single structural change: the monetary regime.
All other parameters (firm heterogeneity, sector calibration, network topology, fiscal policy) remain identical.

\subsection{PLN Regime (Baseline)}

The NBP sets the reference rate using a Taylor rule:
\begin{equation}
    r_t^{\text{NBP}} = \rho \cdot r_{t-1} + (1-\rho)\left[ r^* + \alpha \max(0, \pi_t - \bar\pi) + \beta \max(0, \Delta e_t) \right]
\end{equation}
where $\rho = 0.70$ (inertia), $r^* = 4\%$ (neutral), $\alpha = 1.5$, $\beta = 0.8$, $\bar\pi = 2.5\%$.
The exchange rate adjusts endogenously via balance-of-payments:
\begin{equation}
    \Delta e_t = -\gamma \cdot \frac{\text{BoP}_t}{\text{GDP}_t}, \quad \gamma = 0.02
\end{equation}

\subsection{EUR Regime (Counterfactual)}

The ECB sets rates based on Eurozone-wide inflation $\pi^{\text{EZ}}$, which is exogenous to Poland:
\begin{equation}
    r_t^{\text{ECB}} = \rho^{\text{ECB}} \cdot r_{t-1} + (1-\rho^{\text{ECB}})\left[ r^{*,\text{ECB}} + \alpha^{\text{ECB}} \max(0, \pi^{\text{EZ}} - \bar\pi^{\text{ECB}}) \right]
\end{equation}
where $\rho^{\text{ECB}} = 0.85$, $r^{*,\text{ECB}} = 2.5\%$, $\alpha^{\text{ECB}} = 1.5$, $\bar\pi^{\text{ECB}} = 2.0\%$, $\pi^{\text{EZ}} = 2.0\%$ (constant).

The exchange rate is fixed at the conversion rate (4.33 PLN/EUR), eliminating:
\begin{itemize}
    \item Exchange rate pass-through to inflation
    \item Interest rate parity capital flows
    \item Export competitiveness via depreciation
\end{itemize}

\subsection{Fiscal Constraint (SGP)}

Eurozone membership imposes the Stability and Growth Pact (SGP), which constrains fiscal policy.
In the PLN regime, the government faces no hard fiscal limit --- as a sovereign currency issuer, it can deficit-spend without market-imposed borrowing constraints (we abstract from bond market dynamics).
In the EUR regime, we model two Maastricht criteria as binding constraints on UBI spending:

\begin{enumerate}
    \item \textbf{Flow constraint}: The monthly government deficit must not exceed $3\%/12$ of annual GDP.
    If the planned UBI spending would breach this limit, the effective per-capita UBI is reduced to the maximum fiscally feasible level:
    \begin{equation}
        b_t^{\text{eff}} = \min\!\left(b,\; \frac{\tfrac{0.03}{12} \cdot Y_t^{\text{annual}} + R_{t-1} - G_t^{\text{base}}}{N}\right)
    \end{equation}
    where $b$ is the legislated UBI, $Y_t^{\text{annual}} = 12 \cdot \text{GDP}_{t-1}$, $R_{t-1}$ is previous-period tax revenue, $G_t^{\text{base}}$ is baseline government spending, and $N$ is total population.

    \item \textbf{Stock constraint}: If cumulative public debt exceeds $60\%$ of annual GDP, a progressive austerity multiplier is applied:
    \begin{equation}
        b_t^{\text{eff}} \leftarrow b_t^{\text{eff}} \cdot \max\!\left(0,\; 1 - \kappa\left(\frac{D_t}{Y_t^{\text{annual}}} - 0.60\right)\right), \quad \kappa = 2.0
    \end{equation}
    This models the Excessive Deficit Procedure (EDP), where the European Commission mandates corrective action when debt ratios breach the threshold.
\end{enumerate}

Under PLN, neither constraint applies: the government funds UBI through deficit spending, with the NBP indirectly absorbing the fiscal expansion via its monetary response.

\subsection{Mechanism Hypothesis}

In the PLN regime, UBI triggers a feedback loop:
$\text{UBI} \to \pi\!\uparrow \to r^{\text{NBP}}\!\uparrow \to \text{credit}\!\downarrow \to \text{automation}\!\uparrow \to \text{tech deflation}$.

In the EUR regime, this loop is broken: Polish inflation does not affect the ECB rate, so the credit channel remains loose regardless of domestic conditions.
We hypothesize that this fundamentally alters the system's phase portrait.

% ═══════════════════════════════════════════════════════════════
\section{Results}
\label{sec:results}

We ran 1,260 simulations: 21 UBI levels (0--5,000~PLN in 250~PLN increments) $\times$ 2 regimes $\times$ 30 random seeds.
All simulations share identical firm populations, sector calibration, and network topology; only the monetary regime differs.

\subsection{Regime Comparison at Key UBI Levels}

Table~\ref{tab:regime_comparison} summarizes terminal values (month 120) at three diagnostic UBI levels.

\begin{table}[H]
\centering
\caption{Terminal values at M120: PLN vs EUR (mean $\pm$ std, $N=30$ seeds)}
\label{tab:regime_comparison}
\begin{tabular}{llrrr}
\toprule
\textbf{Metric} & \textbf{Regime} & \textbf{UBI = 0} & \textbf{UBI = 2,000} & \textbf{UBI = 3,000} \\
\midrule
Adoption (\%)     & PLN & $13.7 \pm 2.9$ & $13.0 \pm 1.4$ & $12.1 \pm 1.3$ \\
                  & EUR & $13.4 \pm 2.3$ & $19.1 \pm 2.3$ & $19.1 \pm 2.3$ \\
Inflation (\%)    & PLN & $-8.7$         & $+18.0$        & $+33.8$ \\
                  & EUR & $-23.4$        & $-25.2$        & $-25.2$ \\
Unemployment (\%) & PLN & $8.3$          & $8.5$          & $8.1$ \\
                  & EUR & $70.4$         & $78.4$         & $78.4$ \\
Eff.\ UBI (PLN)   & PLN & $0$            & $2{,}000$      & $3{,}000$ \\
                  & EUR & $0$            & ${\sim}10$     & ${\sim}10$ \\
\bottomrule
\end{tabular}
\end{table}

At UBI$=0$, adoption rates are nearly identical (${\sim}13\%$), confirming that the divergence is driven by the monetary response to fiscal stimulus, not by baseline conditions.
As UBI rises, the regimes diverge sharply.

Under \textbf{PLN}, the NBP raises the reference rate aggressively in response to UBI-driven inflation.
At UBI$=2{,}000$, inflation reaches $+18\%$ and the policy rate tightens accordingly.
This stabilizes employment (${\sim}8.5\%$ unemployment) but prevents further automation: adoption remains flat at ${\sim}13\%$ regardless of UBI level.
The acceleration paradox is effectively suppressed by the monetary tightening channel.

Under \textbf{EUR}, the ECB rate is invariant to Polish conditions, and the SGP fiscal constraint caps UBI spending.
During the first months after UBI activation (month~30), the government delivers the full legislated UBI while the pre-shock fiscal space permits.
As the economy deteriorates (falling GDP, rising unemployment, shrinking tax base), the SGP progressively reduces the effective UBI until it reaches near-zero by month~120.
The result is a \emph{transient stimulus followed by forced austerity}: adoption reaches ${\sim}19\%$ (above the no-UBI baseline of $13\%$) thanks to the initial months of UBI support, but cannot advance further.
Inflation locks at $-25\%$ and unemployment stabilizes at $78\%$.
Crucially, all legislated UBI levels $\geq 1{,}250$~PLN produce \emph{identical outcomes}: the SGP caps effective first-month UBI at ${\sim}1{,}200$~PLN regardless of the legislated amount, and the economy converges to the same equilibrium.

\subsection{Bifurcation Diagram Comparison}

Figure~\ref{fig:bifurcation} overlays the bifurcation diagrams for both regimes.

\begin{figure}[H]
    \centering
    \includegraphics[width=\textwidth]{p02_bifurcation_comparison.png}
    \caption{Bifurcation diagram: PLN vs EUR across 21 UBI levels (30 seeds per point).
    Panel~A: adoption; Panel~B: inflation; Panel~C: critical point signature ($\sigma$ of adoption);
    Panel~D: unemployment. The PLN regime shows a non-monotonic adoption peak at low UBI and rapid inflation rise; the EUR regime shows monotonic adoption growth with persistent deflation and mass unemployment.}
    \label{fig:bifurcation}
\end{figure}

The PLN adoption curve (Panel~A) exhibits a sharp peak at UBI$=250$~PLN (31.5\% adoption, driven by a transient margin squeeze before inflation fully develops), then declines monotonically as the NBP rate tightening dominates.
The EUR adoption curve rises from $13\%$ to $19\%$, then plateaus---the SGP fiscal ceiling prevents further UBI delivery regardless of the legislated amount.

The inflation panel (B) reveals the starkest contrast: PLN inflation rises linearly from $-9\%$ to $+62\%$ across the UBI sweep, while EUR inflation remains locked near $-25\%$---a \emph{deflation trap} that the SGP-constrained fiscal policy cannot break.

\subsection{Critical Point Shift}

Panel~C of Figure~\ref{fig:bifurcation} plots the standard deviation of adoption across seeds as a function of UBI---the signature of a phase transition.
The critical point (maximum variance) occurs at:
\begin{itemize}
    \item \textbf{PLN}: UBI $= 500$~PLN ($\sigma_{\text{adopt}} = 6.4\%$)
    \item \textbf{EUR}: UBI $= 1{,}000$~PLN ($\sigma_{\text{adopt}} = 2.8\%$)
\end{itemize}

The critical point shifts rightward by a factor of two under Eurozone membership, and the maximum variance is substantially lower ($2.8\%$ vs $6.4\%$).
The reduced variance reflects the SGP's smoothing effect: the fiscal constraint prevents the large UBI impulses that generate high adoption variance under PLN, compressing the EUR system into a narrow outcome band.

\subsection{Welfare Analysis}

Figure~\ref{fig:welfare} compares welfare outcomes across regimes.

\begin{figure}[H]
    \centering
    \includegraphics[width=\textwidth]{p02_welfare_comparison.png}
    \caption{Welfare analysis: PLN vs EUR.
    Panel~A: real consumption per capita; Panel~B: Gini coefficient;
    Panel~C: equity--efficiency frontier (lower-right is better);
    Panel~D: key metrics at UBI$=2{,}000$.}
    \label{fig:welfare}
\end{figure}

Real consumption per capita (Panel~A) reveals a stark finding.
Under PLN, optimal consumption occurs at UBI$=500$~PLN (6,159~PLN/capita), declining as inflation erodes purchasing power at higher UBI levels.
Under EUR, the SGP constraint means that by month~120, effective UBI has been cut to ${\sim}10$~PLN.
EUR real consumption is highest at UBI$=0$ (3,778~PLN/capita, where no fiscal deterioration occurs) and \emph{declines} as legislated UBI rises, because the transient UBI burst accelerates the deflationary spiral.
For UBI $\geq 1{,}500$~PLN, EUR consumption flattens at 2,783~PLN/capita---reflecting only wage income of the employed minority.
PLN dominates EUR on consumption for all UBI $\leq 1{,}250$~PLN; above that threshold, PLN's inflation erosion pushes its consumption below the EUR floor.

The Gini coefficient (Panel~B) confirms extreme inequality under EUR.
The EUR Gini is 0.70 at UBI$=0$ and rises to 0.78 for all UBI $\geq 1{,}250$~PLN, reflecting the binary income distribution between employed (wage only, since effective UBI$\to 0$) and unemployed (zero income) populations.
The PLN regime achieves far lower inequality (Gini $< 0.10$ for UBI $\geq 1{,}000$) thanks to near-full employment and sustained UBI delivery.

The equity--efficiency frontier (Panel~C) shows that PLN dominates EUR on equality at every UBI level, with Gini differences exceeding 0.60~pp.
At the PLN optimum (UBI$=500$, cons$=6{,}159$, Gini$=0.18$), PLN achieves twice the consumption and one-quarter the inequality of the best EUR outcome.

\begin{figure}[H]
    \centering
    \includegraphics[width=\textwidth]{p02_regime_difference.png}
    \caption{EUR minus PLN difference across UBI sweep (percentage points).
    Green bars indicate EUR $>$ PLN; red bars indicate EUR $<$ PLN.}
    \label{fig:difference}
\end{figure}

Figure~\ref{fig:difference} summarizes the regime difference.
At UBI$=0$, the regimes are nearly identical.
As UBI rises, EUR adoption exceeds PLN by ${\sim}6$~pp (owing to the transient initial UBI burst), but EUR unemployment exceeds PLN by ${\sim}70$~pp and inflation is $40$--$90$~pp lower (deeper deflation).
Beyond UBI$=1{,}250$~PLN, the EUR curves flatten completely---the SGP has capped effective spending---while PLN inflation continues to rise linearly.

% ═══════════════════════════════════════════════════════════════
\section{Discussion}
\label{sec:discussion}

\subsection{Mundell's Impossible Trinity}

Our results provide a concrete illustration of Mundell's impossible trinity \autocite{mundell1963capital, fleming1962domestic}.
Poland under PLN retains two vertices of the triangle: independent monetary policy (NBP Taylor rule) and free capital flows (interest rate parity channel).
The cost is exchange rate volatility---the PLN depreciates under trade deficits, generating import-push inflation.

Eurozone membership trades independent monetary policy for a fixed exchange rate.
In our simulations, this exchange has dramatic consequences.
The NBP's rate response is the \emph{primary stabilizer} of the labor market: by tightening credit in response to UBI-driven inflation, it prevents the deflationary spiral that destroys firm revenue.
When this stabilizer is removed (EUR regime), the economy loses its ability to self-correct.
Polish inflation diverges from the ECB's target, but the ECB---calibrated to Eurozone-wide conditions---does not respond.

The result is a \textbf{monetary policy mismatch}: Polish domestic conditions require tightening (or at minimum, a neutral response calibrated to local inflation), but the ECB provides a rate calibrated to aggregate Eurozone conditions.
This mismatch is precisely the mechanism identified by \autocite{mundell1961theory} as the cost of non-optimal currency areas.

\subsection{Internal Devaluation Trap}

Under the EUR regime, the economy enters a permanent deflation of ${\sim}25\%$ per annum---far exceeding the soft Bewley floor (\autocite{bewley1999wages}).
This constitutes a classical \textbf{internal devaluation trap} \autocite{schmittgrohe2016downward}: without exchange rate flexibility, adjustment must occur through wages and prices.

The mechanism operates as follows.
Technology-driven deflation (from the $13$--$19\%$ of firms that automate) reduces the price level.
Under PLN, this would be partially offset by exchange rate depreciation (import-push inflation) and NBP rate cuts.
Under EUR, neither channel operates.
The deflation reduces nominal firm revenue, triggering bankruptcies and unemployment.
This is a Fisher debt-deflation spiral \autocite{fisher1933debt} operating at the firm level: falling revenue $\to$ rising real debt burden $\to$ bankruptcy $\to$ further demand contraction.

The SGP fiscal constraint creates a vicious cycle that amplifies this trap.
As the economy contracts, GDP falls and the tax base shrinks, which tightens the $3\%$ deficit ceiling in absolute terms.
Simultaneously, the rising debt-to-GDP ratio triggers the stock constraint's austerity multiplier.
The result is that even a legislated UBI of 2,000~PLN/month is progressively reduced to ${\sim}10$~PLN by month~120---effectively zero.
The government \emph{cannot} sustain the fiscal stimulus needed to break the deflation cycle while remaining within Maastricht criteria.

\subsection{Two Modes of Technological Transition}

The results reveal two qualitatively different pathways for AI-driven structural transformation:

\begin{enumerate}
    \item \textbf{PLN pathway (employment-preserving)}: The NBP's monetary response stabilizes employment but suppresses the adoption incentive.
    Adoption peaks briefly at low UBI levels (the transient acceleration paradox) before being dampened by rate tightening.
    The economy retains near-full employment but undergoes minimal technological transformation.

    \item \textbf{EUR pathway (constrained transformation)}: Without monetary stabilization, the initial UBI impulse briefly stimulates adoption.
    However, the SGP fiscal constraint rapidly eliminates the UBI transfer as the economy deteriorates.
    Adoption reaches ${\sim}19\%$---above PLN's $13\%$, but far below the $55\%$ that a fiscally unconstrained EUR regime would produce.
    The cost is mass structural unemployment ($78\%$), permanent deflation ($-25\%$), and near-zero effective UBI by month~120.
\end{enumerate}

The PLN pathway is unambiguously superior on employment and equality.
The EUR pathway achieves modestly higher adoption ($+6$~pp) at catastrophic social cost.
Unlike an unconstrained EUR scenario where the tradeoff might be debatable, the SGP-constrained EUR regime combines the worst of both worlds: the deflationary trap of a fixed exchange rate \emph{plus} the fiscal straitjacket that prevents compensatory transfers.
This finding strengthens the case for monetary sovereignty during structural transitions.

\subsection{Policy Implications}

These findings carry direct implications for the Polish Euro adoption debate \autocite{borowski2004costs}:

\begin{enumerate}
    \item \textbf{Monetary sovereignty as insurance}: For economies facing potential AI-driven structural shocks, an independent central bank provides a critical stabilization mechanism.
    The NBP's Taylor rule---even without being specifically designed for automation shocks---prevents the deflationary spiral that devastates the EUR scenario.

    \item \textbf{Fiscal-monetary coordination}: Under PLN, UBI effectiveness is limited by the monetary offset (inflation erosion).
    This suggests that UBI policy in a sovereign-currency economy requires explicit coordination with central bank forward guidance to prevent complete fiscal-monetary cancellation.

    \item \textbf{Eurozone entry timing}: If Poland were to join the Eurozone \emph{before} the AI transition, it would face the EUR pathway---modest adoption gains at the cost of devastating unemployment and fiscal collapse.
    If it joins \emph{after} the transition is complete (under PLN's smoother path), the structural adjustment would already be absorbed.
    This argues for sequencing: complete the technological transition under monetary sovereignty, then consider currency union.
\end{enumerate}

\subsection{Limitations}

Several simplifications warrant discussion:

\begin{enumerate}
    \item \textbf{SGP enforcement}: Our model implements the Stability and Growth Pact as a hard constraint on UBI spending.
    In practice, SGP enforcement is politically negotiated---the European Commission has repeatedly granted exemptions (e.g., during COVID-19, under the General Escape Clause activated in 2020--2024).
    Our results therefore represent a ``strict SGP'' scenario; a more lenient interpretation would yield results between the constrained EUR and unconstrained PLN outcomes.

    \item \textbf{No EU transfer mechanisms}: We omit the European Structural and Investment Funds, NextGenerationEU, and potential future fiscal union mechanisms that could partially substitute for national UBI spending.
    In a more complete model, EU-level transfers could mitigate the fiscal constraint.

    \item \textbf{Static ECB}: Eurozone-wide inflation is held constant at $2.0\%$.
    In reality, a major Polish UBI program under EUR could spill over into Eurozone inflation (Poland accounts for ${\sim}3.8\%$ of EU GDP), though the effect would be modest.

    \item \textbf{No bond market dynamics}: The model abstracts from sovereign bond spreads.
    Under EUR without monetary sovereignty, rising debt-to-GDP ratios could trigger market panic (as in the 2010--2012 Eurozone crisis), imposing even tighter fiscal constraints than our SGP implementation.
    Conversely, under PLN, the NBP's implicit backstop prevents self-fulfilling debt crises.

    \item \textbf{Labor mobility}: Eurozone membership enables free labor movement.
    Mass unemployment under EUR could trigger emigration, reducing both the fiscal burden and the domestic labor supply.
    This ``safety valve'' is absent from the model.
\end{enumerate}

% ═══════════════════════════════════════════════════════════════
\section{Conclusion}
\label{sec:conclusion}

This paper tested whether the acceleration paradox identified in \autocite{maciaszek2026acceleration} requires monetary sovereignty.
Using 1,260 Monte Carlo simulations comparing PLN and EUR monetary regimes under Stability and Growth Pact fiscal constraints, we find that it does---decisively.

Under PLN, the acceleration paradox is \emph{transient}: NBP rate tightening suppresses it beyond low UBI levels, but the economy retains near-full employment and policy flexibility.
Under EUR with SGP constraints, the paradox is \emph{stillborn}: a brief UBI impulse raises adoption from $13\%$ to $19\%$, but the SGP's deficit and debt limits progressively eliminate UBI spending as the economy deteriorates.
By month~120, effective UBI falls to ${\sim}10$~PLN regardless of the legislated amount, and the economy is trapped in a permanent deflation ($-25\%$) with mass unemployment ($78\%$).
The critical point shifts from UBI$=500$ (PLN) to UBI$=1{,}000$ (EUR), confirming that monetary regime fundamentally reshapes the system's phase portrait.

The welfare analysis reveals PLN as the dominant regime on employment and equality, while EUR's modest adoption advantage ($+6$~pp) comes at catastrophic social cost.
The SGP fiscal constraint, designed to ensure macroeconomic stability, paradoxically eliminates the fiscal space needed to manage structural transitions---a finding that formalizes the role of Mundell's impossible trinity in the political economy of AI-driven structural change.


% ═══════════════════════════════════════════════════════════════
\printbibliography

\end{document}
